%% sns.tex
%% V2.0
%% 2009/10/11
%% by Norsyidah Mat Saat @ Abas
%% for requirement of Client Server Computing class
%%
%% This is a review paper on Social Networking Services (SNSs). Previously did a paper on eKL but decided to change topics after considering Mr. Zamri's points and given some serious thought!!!

\documentclass[10pt, conference, compsoc]{IEEEtran}
% Add the compsoc option for Computer Society conferences.

\usepackage{graphicx}
\usepackage{url}

\begin{document}
%
% paper title
% can use linebreaks \\ within to get better formatting as desired
\title{THE EMERGENCE OF ONLINE SOCIAL NETWORK SERVICES (SNSs)}


% author names and affiliations

\author{\IEEEauthorblockN{Norsyidah binti Mat Saat @ Abas}
\IEEEauthorblockA{(Matric No.: P50086) \\
Faculty of Technology and Information Sciences  \\
National University of Malaysia (UKM)\\
Bangi, Selangor, Malaysia \\
Email: aznor98@yahoo.com \\
Github: git://github.com/aznor98/latex.git}
}

% make the title area
\maketitle

\begin{abstract}
This paper is to study the emergence of online social networking services (SNSs) like MySpace and Facebook. Besides looking into what Social Networking Site is all about and the historical background of its emergence, this paper also highlights the idea behind the social networking by looking into some interesting theories and studies regarding the social network analysis like the Milgram's Six-Degrees of Separation Theory, the Strength of Weak Ties Theory, and several others. This paper also associates SNSs with some relevant terms like Web 2.0. The relationship between SNSs and Web 2.0 is described. As SNSs has wide usage over the Internet, we also explore on the application of SNSs to the public service domain. We also look into the SNSs stands in the Asia and Malaysia context. Looking at the current trends, we try to predict what future might brings to the development of SNSs. 

\end{abstract}

\section{Introduction}
% no \IEEEPARstart
Besides blogging, Facebook, MySpace, Twitter, LinkedIn, Friendster, Tagged and many other similar online Social Network Services (SNSs) are currently the �hips� of the internet-age nowadays. Many of us have either used the service or at least heard about it. Some of us may have account in more than one SNSs mentioned earlier. Undoubtedly, it is a new form of interacting with other people over the Internet besides the conventional way like emails or instant messaging services. Not too much to say that it has become an emergent phenomenon, not only in the world of Information Technology but also in our lives.  Some of us now might feel something is missing in their lives without entering the SNSs world. A friend of mine has made it a habit to post whatever he is doing on Facebook, either it be what he is eating for breakfast, lunch or dinner or what his thought might be after having an argument with his wife! Besides the famous MySpace and Facebook, other notable SNSs are shown in Figure 1 and 2. \\
\begin{figure}
\includegraphics[width=2.5in]{exampleSNS.jpg}
\caption{Example of Social Networking Services (SNSs)}
\end{figure}

\begin{figure}
\includegraphics[width=3.7in]{popularSNS.jpg}
\caption{The Popular Social Networking Sites (SNSs) and Its Description}
\end{figure}

\subsection{What Is Social Networking Services (SNSs)?}
Social Networking Services or Social Media Sites or Social Web actually refer to the same thing; the online communities who share interests and/or activities, interconnected via the  Internet, which enable them to communicate and share information virtually. It enables users to enhance their networking capabilities by connecting with people in real time. It allows a user to create and maintain an online network of close friends or business associates for personal and professional reasons. 

It is a web-based service that allows individuals to:
(1) construct a public or semi-public profile within a bounded system;
(2) articulate a list of other users with whom they share a connection; and 
(3) view and traverse their list of connections and those made by others within the system.

\subsection{Historical Development of SNSs}
From the beginning, the Internet was a medium for connecting, not only machines but also people. The idea behind SNSs is to make the real-world relationships among people explicitly defined online. The historical development of SNSs shall be illustrated in Figure 3. \\
\begin{figure}
\includegraphics[width=3.7in]{history1.jpg}
\caption{The Historical Development of Significant SNSs}
\end{figure}

\subsection{Features of SNSs}
The backbone of a SNS consists of visible profiles that display an articulated list of Friends who are also users of the system. After joining an SNS, an individual is asked to fill out forms containing a series of questions. The profile is generated using the answers to these questions, which typically include descriptors such as age, location, interests, and an "about me" section. Most sites also encourage users to upload a profile photo. Some sites allow users to enhance their profiles by adding multimedia content or modifying their profile's look and feel. Others, such as Facebook, allow users to add modules ("Applications") that enhance their profile. 

The visibility of a profile varies by site and according to user discretion. By default, profiles on Friendster and Tribe.net are crawled by search engines, making them visible to anyone, regardless of whether or not the viewer has an account. Alternatively, LinkedIn controls what a viewer may see based on whether she or he has a paid account. Sites like MySpace allow users to choose whether they want their profile to be public or "Friends only." Facebook takes a different approach�by default, users who are part of the same "network" can view each other's profiles, unless a profile owner has decided to deny permission to those in their network. Structural variations around visibility and access are one of the primary ways that SNSs differentiate themselves from each other. 

After joining a social network site, users are prompted to identify others in the system with whom they have a relationship. The label for these relationships differs depending on the site�popular terms include "Friends," "Contacts," and "Fans." Most SNSs require bi-directional confirmation for Friendship, but some do not. These one-directional ties are sometimes labeled as "Fans" or "Followers," but many sites call these Friends as well. The term "Friends" can be misleading, because the connection does not necessarily mean friendship in the everyday vernacular sense, and the reasons people connect are varied.

The public display of connections is a crucial component of SNSs. The Friends list contains links to each Friend's profile, enabling viewers to traverse the network graph by clicking through the Friends lists. On most sites, the list of Friends is visible to anyone who is permitted to view the profile, although there are exceptions. For instance, some MySpace users have hacked their profiles to hide the Friends display, and LinkedIn allows users to opt out of displaying their network. 

Most SNSs also provide a mechanism for users to leave messages on their Friends' profiles. This feature typically involves leaving "comments," although sites employ various labels for this feature. In addition, SNSs often have a private messaging feature similar to webmail. While both private messages and comments are popular on most of the major SNSs, they are not universally available.
 
Not all social network sites began as such. QQ started as a Chinese instant messaging service, LunarStorm as a community site, Cyworld as a Korean discussion forum tool, and Skyrock (formerly Skyblog) was a French blogging service before adding SNS features. Classmates.com, a directory of school affiliates launched in 1995, began supporting articulated lists of Friends after SNSs became popular. AsianAvenue, MiGente, and BlackPlanet were early popular ethnic community sites with limited Friends functionality before re-launching in 2005-2006 with SNS features and structure.

Beyond profiles, Friends, comments, and private messaging, SNSs vary greatly in their features and user base. Some have photo-sharing or video-sharing capabilities; others have built-in blogging and instant messaging technology. There are mobile-specific SNSs (e.g., Dodgeball), but some web-based SNSs also support limited mobile interactions (e.g., Facebook, MySpace, and Cyworld). Many SNSs target people from specific geographical regions or linguistic groups, although this does not always determine the site's constituency. Orkut, for example, was launched in the United States with an English-only interface, but Portuguese-speaking Brazilians quickly became the dominant user group (Kopytoff, 2004). Some sites are designed with specific ethnic, religious, sexual orientation, political, or other identity-driven categories in mind. There are even SNSs for dogs (Dogster) and cats (Catster), although their owners must manage their profiles.

All in all, we can summarize the common features in social networking sites as:
\begin{enumerate}
\item Upload pictures;
\item Create profile; 
\item Can be "friends" with other users; 
\item Approval of friend request; and
\item Privacy controls.\\
\end{enumerate}

The other features of SNSs are as follows:
\begin{enumerate}
\item Network of friends (inner circle);
\item Person surfing;
\item Private messaging;
\item Discussion forums;
\item Events management;
\item Blogging and commenting; 
\item Other media uploading; and 
\item Ability to create group with same interests.\\
\end{enumerate}

\section{The Idea Behind SNSs}
When we talked about Social Network, we should not leave out the interesting part of social network analysis. Jacob L. Moreno is one of the founders of the discipline of social network analysis, the branch of sociology that deals with the quantitative evaluation of an individual's role in a group or community by analysis of the network of connections between them and others. His 1934 book Who Shall Survive? contains some of the earliest graphical depictions of social networks. In his studies of the relationship between social structures and psychological well-being, he developed the Sociometry, quantitative method for measuring social relationships. One of his contribution is the Sociogram, a systematic method for graphically representing individuals as points/nodes and the relationships between them as lines/arcs. According to Moreno, the true measure of a social network is the relationship between the nodes. To quote Moreno: "there is a deep discrepancy between the official and the secret behavior of members". Moreno advocates that before any "social program" can be proposed, the sociometrist has to "take into account the actual constitution of the group." What he suggested was people will behave differently among different circle of friends. In other words, how we behave in front of our workmates or supervisors might be different compared to when we're socializing with our former classmates or closest friends.

In Social Network Analysis (SNA), there are technical terms like node, tie, tie-strength and network density. Here are some definition for the terms:
\begin{enumerate}
\item Node: an entity in a network (be it a person, system or group organization;
\item Tie: the relationship or interaction between 2 nodes;
\item Tie-strength: how often and well the 2 nodes relate; and
\item Network density: the number of actual ties compared to the number of possible ties. \\

Besides the Sociometrics by Moreno, there are several other theories relating to Social Network as listed below. 

\subsection{The Six Degrees of Separation Theory}

We all live in a social net of friends, family, workmates, fellow students, acquaintances, etc. Everyone is connected, at least by a friend of a friend (the FOAF concept). There is a theory that anybody is connected to everybody else (on average) by no more than six degrees of separation. 

The Six Degrees of Separation Theory is founded by a sociologist, Stanley Milgram. He conducted an interesting experiment about sending a letter to the target without sending it directly to them. Random people from Nebraska were to send a letter (via intermediaries) to a stock broker in Boston. They could only send to someone with whom they were on first name basis. Among the letters that found the target, the average number of links was six. 

Inspired by the theory, there was a movie in 1993 titled "Six Degrees of Separation", played by Will Smith. There was a famous line in the movie: "I read somewhere that everybody on this planet is separated by only 6 other people. Six Degrees of Separation between us and everyone else on this planet. The President of U.S, a gondolier in Venice, just fill in the name... it's not just big names - it's anyone! A native in a rain forest, a Tiero del Fuegan, an Eskimo. I am bound - you are bound - to everyone on this planet by a trail of 6 people!"

\subsection{The Strength of Weak Ties}

The argument asserts that our acquaintances or the weak ties are likely to be socially involved with one another than are our close friends or the strong ties. Thus the set of people made up of any individual and his or her acquaintances comprises a low-density network, one in which many of the possible relational lines are absent whereas the set consisting of the same individual and his or her close friends will be densely knit where many of the possible lines are present. 

For instance, Ali, Budin and Chua are all in one circle of friend network. Ali, Budin and Chua are connected to the second circle of friends via Chua, whose wife works with Devi. The relationship between Chua and Devi represents the weak tie because Chua does not know Devi directly. Lets say that Budin is looking for a job and there is no job opening in Ali-Budin-Chua's circle. Chua knows Budin is looking for a job and spoke to his wife who then spoke to Devi who then mentioned it to Farouk who is looking for a programmer like Budin. So Budin gets the job at Farouk's place via the strength of weak ties between Chua and Devi. Figure 4 can better shows how the theory of strength of weak ties works. 

\begin{figure}
\includegraphics[width=3.5in]{weakties.jpg}
\caption{How The Strength of Weak Ties Works}
\end{figure}

\subsection{The Small World Phenomenon}  
I am friends with Mimi. I later found out that Mimi went to same school with my workmates, Dilla and they were good friends! It's a small world after all! 

\subsection{The Kevin Bacon Game}
The Kevin Bacon Game was invented by 3 Albright College students back in 1994: Craig Fass, Brian Turtle and Mike Girelly. The goal is to connect any actor to Kevon Bacon, by linking actors who have acted in the same movie. The "Oracle of Bacon" website \url{oracleofbacon.org} uses IMDB to find the shortest link between any 2 actors. The total  number of actors in database is 893,283. Average path length to Kevin is 2.957 and the actor closest to the center is Rod Steigner (2.68). Rank of Kevin, in terms of closeness to center is 1049th. Most actors are within 3 links of each other.    

\section{SNSs and Web 2.0}

\subsection{What is Web 2.0}

Web 2.0 is a perceived second generation of web-based communities and hosted services. It includes applications such as blogs, wikis, RSS feeds, social networking and falksonomies, which aim to facilitate collaboration and sharing between users. Web 2.0 term was made popular by Tim O'Reilly. 

\subsection{Features of Web 2.0}
The features of Web 2.0 according to O'Reilly:
\begin{enumerate}
\item the web as platform;
\item harnesting collective intelligence;
\item data is the next "Intel Inside";
\item end of software release cycle;
\item lightweight programming models;
\item software above the level of a single device;
\item rich user experience. \\
\end{enumerate}

\subsection{Web 2.0 and SNSs}
SNSs is the application of Web 2.0 concept where it comprises of 4 things:
\begin{enumerate}
\item users;
\item content;
\item tags;
\item comments. \\
\end{enumerate}

\section{Why Some SNSs Fail and Why Some Success?}
The same question posed for why Google is a huge success compared to the traditional search engines. The answer is technological advancement and innovative moves. The oldest search engine was Gopher, comes from the word "Go For", means to go and look for something. Among the first in the market for search engine segment are like Altavista, Lycos and Web Crawler. What does Google has that precede these search engines? Google uses indexing while others use keywords. Google index the relationship or ties and rated the popularity of the index. It evaluates the value of the link by looking at the relationship of that page to other pages. If it is referred to many times by many people on certain subject, the possibility of it being relevant to the subject should be very high. For the purpose of this paper, we only look at 3 popular instances: Friendster, MySpace and Facebook. The market share for these 3 SNSs are as follows: Friendster (90 million users), MySpace (253 million users) and Facebook (309 million users). 

\subsection{The Failing Story of Friendster}
Friendster was designed to compete with Match.com, a profitable online dating site. It was designed to help friends-of-friends meet, based on the assumption that friends-of-friends would make better romantic partners than would strangers. Unfortunately, as its popularity surged, the site encountered technical and social difficulties. Friendster's servers and databases were ill-equipped to handle its rapid growth, and the site faltered regularly, frustrating users who replaced email with Friendster. Furthermore, exponential growth meant a collapse in social contexts: Users had to face their bosses and former classmates alongside their close friends. To complicate matters, Friendster began restricting the activities of its most passionate users. Now, the usage of Friendster is fading in U.S. while in Asia it is still being used but not as elaborately as Facebook.  

\subsection{The Success Stories}
Facebook and MySpace are the top dogs in Social Networking today. The Alexa's Top 500 Sites listed Facebook at number 2 after Google and MySpace at eleventh place. \\ 

1)	MySpace \\

MySpace was begun in 2003 to compete with sites like Friendster, Xanga, and AsianAvenue. After rumors emerged that Friendster would adopt a fee-based system, users posted Friendster messages encouraging people to join alternate SNSs, including Tribe.net and MySpace. Because of this, MySpace was able to grow rapidly by capitalizing on Friendster's alienation of its early adopters. One particularly notable group that encouraged others to switch were indie-rock bands who were expelled from Friendster for failing to comply with profile regulations. MySpace contacted local musicians to see how they could support them. The symbiotic relationship between bands and fans helped MySpace expand beyond former Friendster users. Futhermore, MySpace differentiated itself by regularly adding features based on user and by allowing users to personalize their pages. This "feature" emerged because MySpace did not restrict users from adding HTML into the forms that framed their profiles; a copy/paste code culture emerged on the web to support users in generating unique MySpace backgrounds and layouts. Rather than rejecting underage users, MySpace changed its user policy to allow minors. Then, in July 2005, News Corporation purchased MySpace for 580 Million USD (BBC, 2005), attracting massive media attention. \\

2)	Facebook \\

Unlike previous SNSs, Facebook started to support niche demographics before expanding to a broader audience. It was designed to support distinct college networks only. Facebook began in early 2004 as a Harvard-only SNS. Beginning in September 2005, Facebook expanded to include high school students, professionals inside corporate networks, and, eventually, everyone. Unlike other SNSs, Facebook users are unable to make their full profiles public to all users. Another feature that differentiates Facebook is the ability for outside developers to build "Applications" which allow users to personalize their profiles and perform other tasks, such as compare movie preferences and chart travel histories.

\section{The Application of SNS in Public Service Domain}
SNS is more recently being used by various government agencies. Social networking tools serve as a quick and easy way for the government to get the opinion of the public and the keep the public updated on their activity. In United States, The Center for Disease Control demonstrated the importance of vaccinations on the popular children's site Whyville and the National Oceanic and Atmospheric Administration has a virtual island on Second Life where people can explore underground caves or explore the effects of global warming. Similarly, NASA has taken advantage of a few social networking tools, including Twitter and Flickr. They are using these tools to aid t.he Review of U.S. Human Space Flight Plans Committee, whose goal it is to ensure that the nation is on a vigorous and sustainable path to achieving its boldest aspirations in space.

In Malaysian context, perhaps a further study can be conducted to find the feasibility of having this SNSs associated with government�s systems such as MyGov, eKL or BLESS (Business Licensing Electronic Support System). The government can take the advantage of reaching the public at large and getting direct feedback in an informal mode. 

\section{SNSs in Asia}
\begin{figure}
\includegraphics[width=3.5in]{fbvsasian.jpg}
\caption{Facebook vs Asian SNSs}
\end{figure}
Several social networks in Asian markets such as India, China, Japan and Korea have reached not only a high usage but also a high level of profitability. Services such as QQ (China), Mixi (Japan), Cyworld (Korea) or the mobile-focused service Mobile Game Town by the company DeNA in Japan (which has over 10 million users) are all profitable, setting them apart from their western counterparts. Figure 5 shows the comparison for SNSs among Facebook with Asian-based SNSs: Cyworld, Mixi and QQ.\\

\section{Malaysia's Very Own SNS - ruumz}
ruumz is the first full-fledged online social network in Malaysia with an in-built micro-payment and content and service delivery platform. Unlike most other existing social networks, ruumz provides localised content, unique services and on-ground activities for each of its Asian markets, starting with Malaysia, extending the online experience for its members to the real world. 

\section{Emerging Trends in SNSs}
1)	Social networking between businesses - companies are able to drive traffic to their own online sites while encouraging their consumers and clients to have discussions on how to improve or change products or services. \\

2)	The use of Social Networks in the Science communities - allowing scientific groups to expand their knowledge base and share ideas. \\

3)	As a communication tool between teacher-students - extend classroom discussion to posting assignments, tests and quizzes, to assisting with homework outside of the classroom setting.

\section{What Future Will Bring?}

\begin{figure}
\includegraphics[width=2.5in]{future.jpg}
\caption{How Things Will Develop in SNS}
\end{figure}

1)Networks more into the enterprise. For example, Salesforce.com pulls Facebook profiles into its CRM tools. LinkedIn plug-in for IBM Lotus Notes integrates profile directly into email. \\

2)Leverage Social Relationships to target Ads and Offers. SNSs map explicit relationships and identify influencers who are connected AND share. In business, Media6 maps "network neighbours" based on visits to profile sites via browser cookies. For example: Syidah buys on Dell.com.my. Dell.com.my ads are shown to Syidah's closest friends, without identifying or involving her.  \\

3) The Rise of Personal CPM (Customer Profile Management). Augment page in CPM like influence, number of friends, influence among friends, number of influential friends. \\

4) The Social Algorithm will make privacy and permissions easier to manage. \\
(a)Context make content privacy easier; \\
(b)Signals provide a shorthand for our mental map of relationships; \\
(c)Suggest who might want to see it, example: photo sharing.  \\

5) Mobile Social Networking \\

In most mobile communities, mobile phone users can now create their own profiles, make friends, participate in chat rooms, create chat rooms, hold private conversations, share photos and videos, and share blogs by using their mobile phone. Mobile phone users are basically open to every option that someone sitting on the computer has. Some companies provide wireless services which allow their customers to build their own mobile community and brand it, but one of the most popular wireless services for social networking in North America is Facebook Mobile. Other companies provide new innovative features which extend the social networking experience into the real world.\\

6)	Semantic Social Networking \\

Semantic Social Networking means interconnecting both content and people in a meaningful way. Users connected via a common object, i.e.: their job, their university, hobbies, birthdays, etc. By using agreed-upon semantic formats to describe people, content objects and the connection that binds them all together, SNS can interoperate by appealing to common semantics. Developers are already using semantic technologies to augment the ways in which they create, reuse and link profiles and contents on SNS. In the other direction, object-centered social networks can serve as rich data sources for semantic applications. Semantically Linked Online Communities (SIOC) is used to represents contents and actions in online SNS while FOAF (Friend of A Friend) representing people and network. Figure 6 shows some prediction of how things will develop in SNS.\\

\section{Conclusion}

The emergence of SNS will continue to cause a fundamental shift in the way we interact with others and how advertisers can reach people. The social net will be like air that we breathe. The 3 main ingredients for SNSs to be like air are the identity (who we are), the contacts (who we know) and the activities (what we do). As far as the privacy and security concerns, we have to consider that while new technology carries some risk, the positives will most likely outweigh the negatives. Thus, as Facebook, MySpace or other SNSs becoming as a trend in the modern life nowadays, we shall not replace the real community with the virtual one. It is merely a tool to connect people and picking up with our friends especially from remote distance.  

% conference papers do not normally have an appendix

% trigger a \newpage just before the given reference
% number - used to balance the columns on the last page
% adjust value as needed - may need to be readjusted if
% the document is modified later
%\IEEEtriggeratref{8}
% The "triggered" command can be changed if desired:
% IEEEtriggercmd{\enlargethispage{-5in}}

% references section

% can use a bibliography generated by BibTeX as a .bbl file
% BibTeX documentation can be easily obtained at:
% http://www.ctan.org/tex-archive/biblio/bibtex/contrib/doc/
% The IEEEtran BibTeX style support page is at:
% http://www.michaelshell.org/tex/ieeetran/bibtex/
%\bibliographystyle{IEEEtran}
% argument is your BibTeX string definitions and bibliography database(s)
%\bibliography{IEEEabrv,../bib/paper}
%
% <OR> manually copy in the resultant .bbl file
% set second argument of \begin to the number of references
% (used to reserve space for the reference number labels box)

\begin{thebibliography}{1}

\bibitem{IEEEhowto:boyd}
Boyd, D. M., & Ellison, N. B. (2007). Social network sites: Definition, history, and scholarship. Journal of Computer-Mediated Communication, 13(1), article 11. http://jcmc.indiana.edu/vol13/issue1/boyd.ellison.html 

\bibitem{IEEEhowto:breslin}
Breslin, John G., U. Bojars, Stefan Decker, \emph{The Future of Social Networks On The Internet:The Need for Semantics} \url{www.slideshare.net/Cloud/the-future-of-social-networks-on-the-internet-for-semantics?src=related_normal&rel=526441}

\bibitem{Granovetter}
Granovetter, M., \emph{The Strength of Weak Ties: A Network Theory Revisited}, Sociological Theory, Volume 1 (1983), pp 201-233

\bibitem{IEEEhowto:hassan}
Hassan, N. R. 2009. Using Social Network Analysis to Measure IT-Enabled Business Process Performance. Inf. Sys. Manag. 26, 1 (Dec. 2009), 61-76. DOI= \url{http://dx.doi.org/10.1080/10580530802557762} 

\bibitem{IEEEhowto:ravikumar}
Kumar, R., Novak, J., Tomkins, A., \emph{Structure and Evolution of Online Social Networks}

\bibitem{IEEEhowto:charlene}
Li, C., Altimeter Group, charlene@altimetergroup.com, \url{www.slideshare.net/charleneli/sxsw09-the-future-of-social-networks}

\bibitem{rau}
PLP. Rau, Q. Gao, Y. Ding, \emph{Relationship Between The Level of Intimacy And Lurking In Online Social Network Services}, Computers in Human Behavior(2008), Computers in Human Behavior, 24 (2008) pp 2757�2770

\bibitem{tsai}
Tsai, F.S., etal., \emph{Design and Development of A Mobile Peer-To-Peer Social Networking Application}, ScienceDirect e-journal, Expert Systems with Applications 36(2009), 11077�11087

\bibitem{IEEEhowto:vie}
Vie, S., \emph{Digital Divide 2.0: �Generation M� and Online Social
Networking Sites in the Composition Classroom}, ScienceDirect e-journal, Computers and Composition 25 (2008),9�23

\bibitem{wang}
Wang, B. \emph{Survival And Competition Among Social Networking Websites}, Electron. Comm. Res. Appl. (2009), doi:10.1016/j.elerap.2009.08.002

\bibitem{kim}
W.Kim, etal., \emph{On Social Websites}, Informat.Systems(2009), doi:10.1016/j.is.2009.08.003

\bibitem{wiki}
\url{http://en.wikipedia.org/wiki/List_of_social_networking_websites}

\end{thebibliography}

% that's all folks
\end{document}


